Basert på tilbakemeldinger  fra testen fikk vi først og fremst innsikt i hvordan selve prototypen kunne vært bedre. Men vi velger å ha fokus på det vi bør gjøre annerledes med tanke på selve kalenderapplikasjonen siden prototypen ikke representerer det endelige produktet. For eksempel så hadde testpersonene en del problemer med en litt dårlig satt opp rekkefølge på noen av scenarioene våres med tanke på samhandling mellom to brukere. Dette vil ikke være problem ved normal bruk da brukeren kun har sin egen konto å holde styr på.

Vi ble gjort oppmerksomme på at vi burde hatt med tidspunkt i kalenderoversikten, noe vi definitivt kommer til å ta med i den ferdige applikasjonen. Det vi kunne endret på generelt er å inkludere notifikasjoner om at noen har godtatt møtet man har invitert personer til. 
Generelt hadde vi nok hatt god nytte av å hatt en prototype som lignet mer på et ferdig produkt for å få tilbakemeldinger som er enda mer relevante. Men det er for all del lurt å luke bort en del designfeil allerede ved enklere prototyper for å raskere komme inn på rett spor mot en bra applikasjon. 