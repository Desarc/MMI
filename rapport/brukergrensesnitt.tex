\subsection{Innledning}
Vi har valgt å bruke digitale mockups i stedet for papirprototyper. For å lage mockupsene brukte vi nettapplikasjonen \emph{Moqups} \cite{moqups}. For å knytte sammen mockupsene til en interaktiv prototype brukte vi \emph{InVision} \cite{invision}. Den finnes forøvrig her \cite{invisio}. 

\subsection{Deler av brukergrensesnittet}
\subsubsection{Loginvindu}
Loginvinduet er det første som brukeren møter når han starter kalenderapplikasjonen. Vinduet består av to tekstfelter for brukernavn og passord. Så to knapper for å logge samt glemte passord. Vi har valgt å ikke gå inn på grensesnittet for glemte passord. Dersom brukernavn og passordet stemmer på serveren etter at brukeren har trykket på \emph{Login} så vil vinduet lukkes mens hovedvinduet åpnes. Hvis det ikke stemmer så vil brukeren få beskjed om hva som ikke stemte.

\begin{figure}[H]
\centering
\includegraphics[scale=0.65]{images/login.png}
\caption{Loginvindu}
\label{login_image}
\end{figure}

\subsubsection{Hovedvindu}


\begin{figure}[H]
\centering
\includegraphics[scale=0.65]{images/hovedvindu.png}
\caption{Hovedvindu}
\label{hovedvindu_image}
\end{figure}

\subsubsection{Avtaleviser}
hello

\begin{figure}[H]
\centering
\includegraphics[scale=0.65]{images/avtaleviser.png}
\caption{Avtaleviser}
\label{avtaleviser_image}
\end{figure}

\subsubsection{Romreservasjon}
hello

\begin{figure}[H]
\centering
\includegraphics[scale=0.65]{images/romreservasjon.png}
\caption{Romreservasjon}
\label{romreservasjon_image}
\end{figure}

\subsubsection{Flere kalendere}
hello

\begin{figure}[H]
\centering
\includegraphics[scale=0.5]{images/flerekalendere.png}
\caption{Flerekalendere}
\label{flerekalendere_image}
\end{figure}

\subsubsection{Notifikasjon}
hello

\begin{figure}[H]
\centering
\includegraphics[scale=0.65]{images/notifikasjon_akseptert.png}
\caption{Notifikasjon akseptert}
\label{notifikasjon_akseptert_image}
\end{figure}

\begin{figure}[H]
\centering
\includegraphics[scale=0.65]{images/notifikasjon_kanselert.png}
\caption{Notifikasjon kanselert}
\label{notifikasjon_kanselert_image}
\end{figure}


\subsection{Tilstandsdiagram}
Tilstandsdiagram