Hver testperson fikk utlevert et oppgavesett som skulle l�ses ved hjelp av v�r prototype, se vedlegg [TODO vedlegg]. Settet bestod av seks ulike case, der hvert case var satt sammen av en rekke punkter, som skulle gjennomg�s i gitt rekkef�lge av testpersonen, uten innblanding fra testleder og observat�rer. Dette for � kartlegge i hvilken grad brukergrensesnittet var intuitivt nok til � kunne brukes uten veiledning.

\subsection{Case 1}
	I f�rste case skulle testpersonen logge seg inn p� kalendersystemet som Ola Nordmann, og legge inn en ny avtale. Tre deltakere, kalt Beate, Morten og Finn, skulle inviteres, og m�terom R1 skulle reserveres.

\subsection{Case 2}
	Form�let med andre case var � sjekke om testpersonen forsto varslingssystemet i kalenderapplikasjonen. Testpersonen skulle logge inn som Finn Krogstad, som ble invitert til m�tet fra f�rste case, og avsl� alle nye avtaler. Testpersonen m�tte selv finne ut hvordan varsel om nye avtaler er implementert i prototypen.

\subsection{Case 3}
	I tredje case ble testpersonen, etter � ha logget inn som Ola Nordmann, m�tt av et varsel om at Finn Krogstad hadde avsl�tt m�teinnkallelsen. Tidspunkt for m�tet skulle derfor endres til senere p� dagen. Deretter skulle testpersonen logge inn som Finn Krogstad, og sjekke at det var kommet ny innkalling med oppdatert tidspunkt. M�let med testen var at testpersonen skulle pr�ve � redigere en avtale, og sjekke at eventuelle gjester fikk beskjed om endringene.

\subsection{Case 4}
	Fjerde case startet p� samme m�te som tredje case. Her skulle testpersonen ta bort Finn fra m�teinvitasjoen. Finn Krogstad skulle f� beskjed om at m�tet var avlyst, noe testpersonen skulle bekrefte.

\subsection{Case 5}
	I femte case skulle testpersonen, logget inn som Ola Nordmann, pr�ve � avlyse et m�te. Testpersonen skulle s� bekrefte at Finn Krogstad hadde f�tt beskjed om at m�tet var avlyst.

\subsection{Case 6}
	Sjette case skulle teste funksjonaliteten for � vise andre brukeres kalender sammen med sin egen. Testpersonen, logget inn som Ola Nordmann, skulle f�rst pr�ve � vise kalenderen til Beate, og bekrefte at hennes avtaler ble vist sammen med Ola sine. Deretter skulle Beates kalender skjules.