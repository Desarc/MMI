Hver testperson fikk utlevert et oppgavesett som skulle løses ved hjelp av vår prototype, se vedlegg: \hyperlink{vedlegg/cases.pdf.1}{Cases}. Settet bestod av seks ulike case, der hvert case var satt sammen av en rekke punkter, som skulle gjennomgås i gitt rekkefølge av testpersonen, uten innblanding fra testleder og observatører. Dette for å kartlegge i hvilken grad brukergrensesnittet var intuitivt nok til å kunne brukes uten veiledning.

\subsection{Case 1}
	I første case skulle testpersonen logge seg inn på kalendersystemet som Ola Nordmann, og legge inn en ny avtale. Tre deltakere, kalt Beate, Morten og Finn, skulle inviteres, og møterom R1 skulle reserveres.

\subsection{Case 2}
	Formålet med andre case var å sjekke om testpersonen forsto varslingssystemet i kalenderapplikasjonen. Testpersonen skulle logge inn som Finn Krogstad, som ble invitert til møtet fra første case, og avslå alle nye avtaler. Testpersonen måtte selv finne ut hvordan varsel om nye avtaler er implementert i prototypen.

\subsection{Case 3}
	I tredje case ble testpersonen, etter å ha logget inn som Ola Nordmann, møtt av et varsel om at Finn Krogstad hadde avslått møteinnkallelsen. Tidspunkt for møtet skulle derfor endres til senere på dagen. Deretter skulle testpersonen logge inn som Finn Krogstad, og sjekke at det var kommet ny innkalling med oppdatert tidspunkt. Målet med testen var at testpersonen skulle prøve å redigere en avtale, og sjekke at eventuelle gjester fikk beskjed om endringene.

\subsection{Case 4}
	Fjerde case startet på samme måte som tredje case. Her skulle testpersonen ta bort Finn fra møteinvitasjonen. Finn Krogstad skulle få beskjed om at møtet var avlyst, noe testpersonen skulle bekrefte.

\subsection{Case 5}
	I femte case skulle testpersonen, logget inn som Ola Nordmann, prøve å avlyse et møte. Testpersonen skulle så bekrefte at Finn Krogstad hadde fått beskjed om at møtet var avlyst.

\subsection{Case 6}
	Sjette case skulle teste funksjonaliteten for å vise andre brukeres kalender sammen med sin egen. Testpersonen, logget inn som Ola Nordmann, skulle først prøve å vise kalenderen til Beate, og bekrefte at hennes avtaler ble vist sammen med Ola sine. Deretter skulle Beates kalender skjules.