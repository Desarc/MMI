Vi kjørte to iterasjoner på testen, med to ulike testere. I fra den andre
gruppen var det Merete og så Ove som testet. I  den første iterasjonen var
Kristian leder, Stian og Øyvin var testobservatører. Stian fylte ut et observasjonsskjema hvor det i hovedsak fokuseres på problemer med prototypen. Øyvin tok for seg både positive og negative faktorer ved testing av prototypen vår. I den andre iterasjonen var
Stian leder og Øyvin testobservatør. Kristian og Rune ble testet av den andre
gruppen. Testen ble gjennomført ved at vi startet ved  å hilse på testeren og introduserte oss
kort. Vi forklarte at vi hadde valgt å bruke et digitalt alternativ til papirprototypen. Det var derfor viktig å få frem at selv om det lignet mer på et vanlig dataprogram, så var det fremdeles bare en enkel digital variant av en papirprototype. Deretter gikk vi i gang med å forklare hvordan testen skulle utføres og forklarte hvordan vi ville at testeren skulle tenke høyt og så videre. Etter testen lot vi testeren uttale seg om hvordan det var å teste, og så intervjuet vi dem kort og takket for deltakelsen. 
Vi hadde også laget oss et lite manus som vi benyttet og ligger vedlagt i denne rapporten.


